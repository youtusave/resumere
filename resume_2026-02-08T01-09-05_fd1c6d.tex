\documentclass[letterpaper,11pt]{article}
\usepackage[utf8]{inputenc}
\usepackage{latexsym}
\usepackage[empty]{fullpage}
\usepackage{titlesec}
\usepackage{marvosym}
\usepackage[usenames,dvipsnames]{color}
\usepackage{verbatim}
\usepackage{enumitem}
\usepackage[hidelinks]{hyperref}
\usepackage{fancyhdr}
\usepackage[english]{babel}
\usepackage{tabularx}
\usepackage[default]{sourcesanspro}

% Page styling
\pagestyle{fancy}
\fancyhf{} % Clear all header and footer fields
\renewcommand{\headrulewidth}{0pt} % No header rule
\renewcommand{\footrulewidth}{0pt} % No footer rule
\addtolength{\oddsidemargin}{-0.5in}
\addtolength{\evensidemargin}{-0.5in}
\addtolength{\textwidth}{1in}
\addtolength{\topmargin}{-.5in}
\addtolength{\textheight}{1.0in}
\urlstyle{same}
\raggedbottom
\raggedright
\setlength{\tabcolsep}{0in} % No padding for tabular cells

% Section formatting
\titleformat{\section}{\vspace{-4pt}\scshape\raggedright\large}{}{0em}{}[\color{black}\titlerule \vspace{-5pt}]

% Custom resume commands
\newcommand{\resumeItem}[1]{\item\small{{\#1 \vspace{-2pt}}}}
\newcommand{\resumeSubheading}[4]{
  \vspace{-2pt}
  \item
  \begin{tabular*}{0.97\textwidth}[t]{l@{\extracolsep{\fill}}r}
    \textbf{\#1} & \#2 \\
    \textit{\small\#3} & \textit{\small \#4} \\
  \end{tabular*}
  \vspace{-7pt}
}
\newcommand{\resumeProjectHeading}[2]{
  \item
  \begin{tabular*}{0.97\textwidth}{l@{\extracolsep{\fill}}r}
    \small\#1 & \#2 \\
  \end{tabular*}
  \vspace{-7pt}
}
\newcommand{\resumeSubHeadingListStart}{\begin{itemize}[leftmargin=0.15in, label={}]}
\newcommand{\resumeSubHeadingListEnd}{\end{itemize}}
\newcommand{\resumeItemListStart}{\begin{itemize}}
\newcommand{\resumeItemListEnd}{\end{itemize}\vspace{-5pt}}

\begin{document}

% Header
\begin{center}
  \textbf{\Huge \scshape Your Name} \\
  \vspace{1pt}
  \small 123-456-7890 $|$ \href{mailto:email@domain.com}{\underline{email@domain.com}} $|$ \href{https://www.linkedin.com/in/yourprofile}{\underline{linkedin.com/in/yourprofile}} $|$ \href{https://github.com/yourusername}{\underline{github.com/yourusername}}
\end{center}

% --- Sections ---

% Education
\section{Education}
\resumeSubHeadingListStart
  \resumeSubheading
    {Degree Name} % e.g., Master of Science in Computer Science
    {University Name} % e.g., Stanford University
    {City, State} % e.g., Stanford, CA
    {Month Year} % e.g., May 2023
  \resumeItemListStart
    \resumeItem{GPA: X.XX/4.00}
    \resumeItem{Relevant Coursework: Course 1, Course 2, Course 3}
  \resumeItemListEnd
\resumeSubHeadingListEnd

% Experience
\section{Experience}
\resumeSubHeadingListStart
  \resumeSubheading
    {Your Job Title} % e.g., Software Engineer
    {Company Name} % e.g., Google
    {City, State} % e.g., Mountain View, CA
    {Month Year -- Month Year} % e.g., June 2023 -- Present
  \resumeItemListStart
    \resumeItem{Developed and implemented feature X, resulting in Y\% increase in user engagement.}
    \resumeItem{Collaborated with a team of Z engineers to design and deploy scalable microservices.}
    \resumeItem{Utilized [Technology A] and [Technology B] to optimize database performance by X\% .}
  \resumeItemListEnd
  \resumeSubheading
    {Previous Job Title} % e.g., Intern
    {Previous Company Name} % e.g., Microsoft
    {City, State} % e.g., Redmond, WA
    {Month Year -- Month Year} % e.g., May 2022 -- August 2022
  \resumeItemListStart
    \resumeItem{Assisted in the development of [Project Name] using [Language/Tool].}
    \resumeItem{Conducted research on [Topic] and presented findings to the team.}
  \resumeItemListEnd
\resumeSubHeadingListEnd

% Projects
\section{Projects}
\resumeSubHeadingListStart
  \resumeProjectHeading
    {Project Name 1} % e.g., Personal Portfolio Website
    {Link to Project/GitHub} % e.g., \href{https://github.com/yourusername/project1}{\underline{GitHub}}
  \resumeItemListStart
    \resumeItem{Built a responsive personal portfolio website using HTML, CSS, and JavaScript.}
    \resumeItem{Integrated [Feature] to showcase [Content].}
  \resumeItemListEnd
  \resumeProjectHeading
    {Project Name 2} % e.g., Machine Learning Model for [Task]
    {Link to Project/GitHub} % e.g., \href{https://github.com/yourusername/project2}{\underline{GitHub}}
  \resumeItemListStart
    \resumeItem{Developed a machine learning model using Python and scikit-learn to [Task].}
    \resumeItem{Achieved X\% accuracy on the test dataset.}
  \resumeItemListEnd
\resumeSubHeadingListEnd

% Skills
\section{Skills}
\resumeSubHeadingListStart
  \resumeSubheading
    {Programming Languages:}
    {Python, Java, C++, JavaScript, SQL}
    {} {}
  \resumeSubheading
    {Frameworks/Libraries:}
    {React, Node.js, Django, Flask, TensorFlow, PyTorch, scikit-learn}
    {} {}
  \resumeSubheading
    {Tools/Technologies:}
    {Git, Docker, Kubernetes, AWS, Azure, Jira}
    {} {}
  \resumeSubheading
    {Languages:}
    {English (Native), Spanish (Fluent)}
    {} {}
\resumeSubHeadingListEnd

% Awards and Recognition (Optional)
% \section{Awards and Recognition}
% \resumeSubHeadingListStart
%   \resumeItem{Recipient of the [Award Name], [Year]}
%   \resumeItem{[Scholarship Name] Scholarship, [Year]}
% \resumeSubHeadingListEnd

% --- End of Resume ---
\end{document}


**Explanation and How to Use:**

1.  **Save as `.tex`:** Save this code in a plain text file with a `.tex` extension (e.g., `resume.tex`).
2.  **Compile:** You'll need a LaTeX distribution (like TeX Live or MiKTeX) installed on your computer. You can then compile the `.tex` file using a LaTeX editor (like TeXstudio, VS Code with a LaTeX extension, or Overleaf online) or from the command line:
    bash
    pdflatex resume.tex
    
    You might need to run it twice for the references to resolve correctly.
3.  **Fill in Your Details:**
    *   `\textbf{\Huge \scshape Your Name}`: Replace `Your Name` with your actual name.
    *   `123-456-7890`: Replace with your phone number.
    *   `email@domain.com`: Replace with your email address.
    *   `linkedin.com/in/yourprofile`: Replace with your LinkedIn profile URL.
    *   `github.com/yourusername`: Replace with your GitHub profile URL.
    *   **Education Section:** Fill in your degree, university, location, and graduation date. Add relevant coursework or GPA if desired.
    *   **Experience Section:**
        *   Replace `Your Job Title`, `Company Name`, `City, State`, and dates for each position.
        *   Use `\resumeItem{...}` to list your responsibilities and achievements. **Crucially, use action verbs and quantify your accomplishments whenever possible.**
    *   **Projects Section:**
        *   Replace `Project Name` and provide a link to your project's repository or live demo.
        *   Use `\resumeItem{...}` to describe the project and your role.
    *   **Skills Section:** Customize the categories and list your relevant skills.
    *   **Optional Sections:** Uncomment and fill in the "Awards and Recognition" section if applicable.

**Key Features of the Template:**

*   **Clean Design:** Uses `sourcesanspro` font for a modern look.
*   **Structured Sections:** Clearly defined sections for Education, Experience, Projects, and Skills.
*   **Custom Commands:** `\resumeItem`, `\resumeSubheading`, `\resumeProjectHeading` simplify formatting.
*   **Hyperlinks:** `hyperref` package allows for clickable email and website links.
*   **Page Margins:** `fullpage` package and manual adjustments ensure good use of space.
*   **No Headers/Footers:** `fancyhdr` is used to clear default headers and footers.

This is a solid foundation. Remember to tailor the content of each bullet point to the specific job you're applying for, highlighting the most relevant skills and experiences. Good luck!