\documentclass[letterpaper,11pt]{article}

\usepackage[utf8]{inputenc}
\usepackage{latexsym}
\usepackage[empty]{fullpage}
\usepackage{titlesec}
\usepackage{marvosym}
\usepackage[usenames,dvipsnames]{color}
\usepackage{verbatim}
\usepackage{enumitem}
\usepackage[hidelinks]{hyperref}
\usepackage{fancyhdr}
\usepackage[english]{babel}
\usepackage{tabularx}
\input{glyphtounicode}


%----------FONT OPTIONS----------
\usepackage[default]{sourcesanspro}


\pagestyle{fancy}
\fancyhf{}
\fancyfoot{}
\renewcommand{\headrulewidth}{0pt}
\renewcommand{\footrulewidth}{0pt}

% Adjust margins
\addtolength{\oddsidemargin}{-0.5in}
\addtolength{\evensidemargin}{-0.5in}
\addtolength{\textwidth}{1in}
\addtolength{\topmargin}{-.5in}
\addtolength{\textheight}{1.0in}

\urlstyle{same}

\raggedbottom
\raggedright
\setlength{\tabcolsep}{0in}

% Sections formatting
\titleformat{\section}{
\vspace{-4pt}\scshape\raggedright\large
}{}{0em}{}[\color{black}\titlerule \vspace{-5pt}]

% Ensure that generate pdf is machine readable/ATS parsable
\pdfgentounicode=1

%-------------------------
% Custom commands
\newcommand{\resumeItem}[1]{
\item\small{
{#1 \vspace{-2pt}}
}
}

\newcommand{\resumeSubheading}[4]{
\vspace{-2pt}\item
\begin{tabular*}{0.97\textwidth}[t]{l@{\extracolsep{\fill}}r}
\textbf{#1} & #2 \\
\textit{\small#3} & \textit{\small #4} \\
\end{tabular*}\vspace{-7pt}
}

\newcommand{\resumeSubSubheading}[2]{
\item
\begin{tabular*}{0.97\textwidth}{l@{\extracolsep{\fill}}r}
\textit{\small#1} & \textit{\small #2} \\
\end{tabular*}\vspace{-7pt}
}

\newcommand{\resumeProjectHeading}[2]{
\item
\begin{tabular*}{0.97\textwidth}{l@{\extracolsep{\fill}}r}
\small#1 & #2 \\
\end{tabular*}\vspace{-7pt}
}

\newcommand{\resumeSubItem}[1]{\resumeItem{#1}\vspace{-4pt}}

\renewcommand\labelitemii{$\vcenter{\hbox{\tiny$\bullet$}}$}

\newcommand{\resumeSubHeadingListStart}{\begin{itemize}[leftmargin=0.15in, label={}]}
\newcommand{\resumeSubHeadingListEnd}{\end{itemize}}
\newcommand{\resumeItemListStart}{\begin{itemize}}
\newcommand{\resumeItemListEnd}{\end{itemize}\vspace{-5pt}}

%-------------------------------------------
%%%%%%  RESUME STARTS HERE  %%%%%%%%%%%%%%%%%%%%%%%%%%%%


\begin{document}

% --- Default Header - AI will replace this based on profile ---
\begin{center}
\textbf{\Huge \scshape Example Name} \\ \vspace{1pt}
\small 123-456-7890 $|$ \href{mailto:example@example.com}{\underline{example@example.com}} $|$
\href{https://linkedin.com/in/example}{\underline{linkedin.com/in/example}} $|$
\href{https://github.com/example}{\underline{github.com/example}}
\end{center}

% --- Default Sections - AI will replace these ---

\section{Professional Summary} % Optional. If the resume is already long, no need for summary. ONLY need when resume is still short
A highly motivated and skilled professional with experience in software development, data analysis, and project management. Proven ability to work independently and collaboratively to deliver high-quality results. Seeking a challenging and rewarding position where I can utilize my skills and contribute to the success of the organization.

\section{Education}
\resumeSubHeadingListStart
\resumeSubheading{Example University}{City, State}{Bachelor of Science in Computer Science}{2020-2024}
% Optional: list coursework in a horizontal format, separate by commas
\resumeSubHeadingListEnd

\section{Experience}
\resumeSubHeadingListStart
\resumeSubheading{Example Company}{2022-2023}{Software Engineering Intern}{City, State}
\resumeItemListStart
\resumeItem{Developed and maintained software applications using Java and Python.}
\resumeItem{Collaborated with a team of engineers to design and implement new features.}
\resumeItemListEnd
\resumeSubHeadingListEnd

\section{Projects}
\resumeSubHeadingListStart
\resumeProjectHeading{\textbf{Example Project} $|$ \emph{Python, Machine Learning}}{2023}
\resumeItemListStart
\resumeItem{Developed a machine learning model to predict customer churn.}
\resumeItem{Achieved 90\% accuracy in predicting customer churn.}
\resumeItemListEnd
\resumeSubHeadingListEnd

\section{Technical Skills} % Customize the format depends on the role
\begin{itemize}[leftmargin=0.15in, label={}]
\item{
\textbf{Languages}{: Java, Python, C++} \\
\textbf{Data Analysis}{: SQL, R, Pandas} \\
\textbf{Project Management}{: Agile, Scrum}
} \\
% Add more rows for topics, or skills if have data, and depends on the industry
\end{itemize}


%-------------------------------------------
\end{document}